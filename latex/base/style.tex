% style sheet for the thesis
% Chapter and section title format
\titleformat{\section}[block]{\normalfont\huge\bfseries}{\thesection.}{.5em}{\Huge}[{}]
% \titlespacing*{\chapter}{0pt}{-19pt}{25pt}
% \titleformat{\section}[block]{\normalfont\Large\bfseries}{\thesection.}{.5em}{\Large}



% code formatting with listing
\lstset{
  basicstyle=\ttfamily,
  breaklines=true,
}

% Margins
\geometry{
    a4paper,
    margin=2.75cm
}

% Index depth limit
\setcounter{tocdepth}{2}

% Paragraph Indentation
\setlength{\parindent}{1cm}

\renewcommand{\lstlistingname}{Code extraction}
\renewcommand*{\lstlistlistingname}{Code Excerpts Index}

\definecolor{US_red}{cmyk}{0, 1, 0.65, 0.34}
\definecolor{US_yellow}{cmyk}{0, 0.3, 0.94, 0}

\mdfdefinestyle{US_style}{backgroundcolor=US_yellow!20, font=\bfseries, hidealllines=true}


\renewcommand{\contentsname}{Sommario}
%\renewcommand{\chaptername}{Macroargomento}

% style for chapter and section names
% see also https://texblog.org/2012/07/03/fancy-latex-chapter-styles/
\usepackage[T1]{fontenc}
\usepackage{titlesec, blindtext, color}
\definecolor{gray75}{gray}{0.75}
\definecolor{gray45}{gray}{0.45}
\definecolor{gray25}{gray}{0.25}
\definecolor{unipd}{HTML}{870011}
\newcommand{\hsp}{\hspace{20pt}}
\titleformat{\chapter}[hang]{\Huge\bfseries}{\textcolor{gray45}{\thechapter}\hsp\textcolor{gray75}{|}\hsp}{0pt}{\Huge\bfseries\textcolor{unipd}}[\color{unipd}{\titlerule[1.8pt]}]

%\usepackage[Bjornstrup]{fncychap}

\newcommand{\resp}[1]{\vspace{-1.5truecm}\noindent\textcolor{unipd}{\textbf{Task leader(s):}}~\emph{#1}}
\newcommand{\authors}[1]{\begin{scriptsize}\textcolor{unipd}{\textbf{Credits:}} #1\end{scriptsize}}
\newcommand{\startsolution}{\vspace{0.35truecm}\noindent\textcolor{unipd}{\textbf{Risoluzione del problema}}\vspace{0.35truecm}}
\newcommand{\solutionpoint}[1]{\noindent\textbf{Q#1.}~}

\pagestyle{fancy}
\renewcommand{\footrulewidth}{0.2pt}% default is 0pt
\renewcommand{\headrulewidth}{0.2pt}% default is 0pt
\fancyhf{}
\rhead{\bfseries \textcolor{unipd}{\emph{PoCN}}}
\lhead{\bfseries \textcolor{unipd}{\thepage}}
\lfoot{\bfseries \footnotesize\textcolor{unipd}{Project Report}}
\rfoot{\bfseries \footnotesize\textcolor{unipd}{\today}}

\newcommand\setItemnumber[1]{\setcounter{enumi}{\numexpr#1-1\relax}}

%\theoremstyle{definition}
%\newtheorem{example}{Esempio}[section]

% boxes
\usepackage{fancybox}
\usepackage[framemethod=tikz]{mdframed}
\usetikzlibrary{calc}
\usepackage{chngcntr}

% counters
\newcounter{theorem}
\newcounter{example}
\newcounter{mdbox}
\counterwithin{theorem}{section}
\counterwithin{example}{section}
\counterwithin{mdbox}{section}


% names for the structures
\newcommand\theoname{Teorema}
\newcommand\examname{Esempio}

\makeatletter
% mdf key for the eventual notes in the structures
\def\mdf@@mynote{}
\define@key{mdf}{mynote}{\def\mdf@@mynote{#1}}

% style for theorems
\mdfdefinestyle{mytheo}{
settings={\refstepcounter{theorem}},
linewidth=1pt,
innertopmargin=1.5\baselineskip,
roundcorner=10pt,
backgroundcolor=blue!05,
linecolor=orange,
singleextra={
  \node[xshift=10pt,thick,draw=blue,fill=blue!20,rounded corners,anchor=west] at (P-|O) %
  {\strut{\bfseries\theoname~\thetheorem}\ifdefempty{\mdf@@mynote}{}{~(\mdf@@mynote)}};
},
firstextra={
  \node[xshift=10pt,thick,draw=blue,fill=blue!20,rounded corners,anchor=west] at (P-|O) %
  {\strut{\bfseries\theoname~\thetheorem}\ifdefempty{\mdf@@mynote}{}{~(\mdf@@mynote)}};
}
}

% style for example
\mdfdefinestyle{myexam}{
settings={\refstepcounter{example}},
linewidth=1pt,
innertopmargin=1.5\baselineskip,
roundcorner=10pt,
backgroundcolor=white,
linecolor=unipd,
singleextra={
  \node[xshift=10pt,thick,draw=unipd,fill=unipd!20,rounded corners,anchor=west] at (P-|O) %
  {\strut{\bfseries\examname~\theexample}\ifdefempty{\mdf@@mynote}{}{~(\mdf@@mynote)}};
},
firstextra={
  \node[xshift=10pt,thick,draw=unipd,fill=unipd!20,rounded corners,anchor=west] at (P-|O) %
  {\strut{\bfseries\examname~\theexample}\ifdefempty{\mdf@@mynote}{}{~(\mdf@@mynote)}};
}
}

% style for my box
\mdfdefinestyle{mybox}{
settings={},
linewidth=1pt,
innertopmargin=1.5\baselineskip,
roundcorner=10pt,
backgroundcolor=white,
linecolor=unipd,
singleextra={
  \node[xshift=10pt,thick,draw=unipd,fill=unipd!20,rounded corners,anchor=west] at (P-|O) %
  {\strut{\bfseries}\ifdefempty{\mdf@@mynote}{}{~\mdf@@mynote}};
},
firstextra={
  \node[xshift=10pt,thick,draw=unipd,fill=unipd!20,rounded corners,anchor=west] at (P-|O) %
  {\strut{\bfseries}\ifdefempty{\mdf@@mynote}{}{~\mdf@@mynote}};
}
}


% some auxiliary environments
\newmdenv[style=mytheo]{theor}
\newmdenv[style=myexam]{exam}
\newmdenv[style=mybox]{mddbox}

% the actual environments
\newenvironment{theorem}[1][]
  {\begin{theor}[mynote=#1]}
  {\end{theor}}
\newenvironment{mdbox}[1][]
  {\begin{mddbox}[mynote=#1]}
  {\end{mddbox}}
\newenvironment{example}[1][]
  {\begin{exam}[mynote=#1]}
  {\end{exam}}

\makeatother